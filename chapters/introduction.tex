\chapter{Introduction}\label{introduction}

This book aims to teach the basics of the Python programming language using a practical approach. Its method is quite basic though: after a very simple introduction to each topic, the reader is invited to learn by solving the proposed exercises.

These exercises have been used extensively in my web development and distributed computing classes at the Superior School of Technology of Setúbal. With these exercises, most students are at full speed with Python in less than a month. In fact, students of the distributed computing course, taught in the second year of the software engineering degree, become familiar with Python's syntax in two weeks and are able to implement a distributed client-server application with sockets in the third week.

This book is divided in the following chapters: in chapter \ref{installation} I will provide the basic installation instructions and execution of the Python interpreter. In chapter \ref{basic-datatypes} we will talk about the most basic data types, numbers and strings. In chapter \ref{functions} we will start tinkering with functions, and in chapter \ref{loops} the topic is about "loops". In chapter \ref{dictionaries} we will work with dictionaries and finally, in chapter \ref{classes} we will finish the book with some exercises about classes and object oriented programming.

Please note that this book is a work in progress and as such may contain quite a few spelling errors that may be corrected in the future. However it is made available as it is so it can be useful to anyone who wants to use it. I sincerely hope you can get something good through it.

This book is made available in github (check it at \url{https://github.com/joaoventura/full-speed-python}) so I appreciate any pull requests to correct misspellings or to suggest new exercises or clarification of the current content.

Best wishes,

João Ventura - Adjunct Professor at the Escola Superior de Tecnologia de Setúbal
