\chapter{Iterators}\label{iterators}

As we saw previously, in Python we use the "for" loop to iterate over the contents of objects:

\begin{lstlisting}
>>> for value in [0, 1, 2, 3, 4, 5]:
...     print(value)
... 
0
1
4
9
16
25
\end{lstlisting}

Objects that can be used with a "for" loop are called iterators. An iterator is, therefore, an object that follows the iteration protocol.

The built-in function "iter" can be used to build iterator objects, while the "next" function can be used to gradually iterate over their content:

\begin{lstlisting}
>>> my_iter = iter([1, 2, 3])
>>> my_iter
<list_iterator object at 0x10ed41cc0>
>>> next(my_iter)
1
>>> next(my_iter)
2
>>> next(my_iter)
3
>>> next(my_iter)
Traceback (most recent call last):
  File "<stdin>", line 1, in <module>
StopIteration
\end{lstlisting}

If there are no more elements, the iterator raises a "StopIteration" exception.

\section{Iterator classes}

Iterators can be implemented as classes. You just need to implement the "\_\_next\_\_" and "\_\_iter\_\_" methods. Here's an example of a class that mimics the "range" function, returning all values from "a" to "b":

\begin{lstlisting}
class MyRange:

    def __init__(self, a, b):
        self.a = a
        self.b = b

    def __iter__(self):
        return self

    def __next__(self):
        if self.a < self.b:
            value = self.a
            self.a += 1
            return value
        else:
            raise StopIteration
\end{lstlisting}

Basically, on every call to "next" it moves forward the internal variable "a" and returns its value. When it reaches "b", it raises the StopIteration exception.

\begin{lstlisting}
>>> myrange = MyRange(1, 4)
>>> next(myrange)
1
>>> next(myrange)
2
>>> next(myrange)
3
>>> next(myrange)
Traceback (most recent call last):
  File "<stdin>", line 1, in <module>
StopIteration
\end{lstlisting}

But most important, you can use the iterator class in a "for" loop:

\begin{lstlisting}
>>> for value in MyRange(1, 4):
...     print(value)
... 
1
2
3
\end{lstlisting}


\section{Exercises with iterators}

\begin{enumerate}

\item Implement an iterator class to return the square of all numbers from "a" to "b".

\item Implement an iterator class to return all the even numbers from 1 to $n$.

\item Implement an iterator class to return all the odd numbers from 1 to $n$.

\item Implement an iterator class to return all numbers from $n$ down to 0.

\item Implement an iterator class to return the fibonnaci sequence from the first element up to $n$. You can check the definition of the fibonnaci sequence in the function's chapter. These are the first numbers of the sequence: 0, 1, 1, 2, 3, 5, 8, 13, 21, 34, 55, 89, ...

\item Implement an iterator class to return all consecutive pairs of numbers from 0 until $n$, such as (0, 1), (1, 2), (2, 3)...

\end{enumerate}